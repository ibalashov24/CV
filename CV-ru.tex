\documentclass[letterpaper,10pt]{article}

\usepackage{latexsym}
\usepackage[empty]{fullpage}
\usepackage{titlesec}
\usepackage{marvosym}
\usepackage[usenames,dvipsnames]{color}
\usepackage{verbatim}
\usepackage{enumitem}
\usepackage[hidelinks]{hyperref}
\usepackage{fancyhdr}
\usepackage{tabularx}
\usepackage{multicol}
\input{glyphtounicode}

% Russian
\usepackage[utf8]{inputenc}
\usepackage[english,main=russian]{babel}

%\usepackage[default]{sourcesanspro}
\usepackage[T1]{fontenc}

\pagestyle{fancy}
\fancyhf{} 
\fancyfoot{}
\renewcommand{\headrulewidth}{0pt}
\renewcommand{\footrulewidth}{0pt}


\addtolength{\oddsidemargin}{-0.5in}
\addtolength{\evensidemargin}{-0.5in}
\addtolength{\textwidth}{1in}
\addtolength{\topmargin}{-.5in}
\addtolength{\textheight}{1.0in}

\urlstyle{same}

\raggedbottom
\raggedright
\setlength{\tabcolsep}{0in}

\titleformat{\section}{
  \vspace{-4pt}\centering
}{}{0em}{}[\color{black}\titlerule\vspace{-5pt}]


\pdfgentounicode=1

\newcommand{\resumeItem}[1]{
  \item\small{
    {#1 \vspace{-2pt}}
  }
}

\newcommand{\resumeSubheading}[4]{
  \vspace{-2pt}\item
    \begin{tabular*}{0.97\textwidth}[t]{l@{\extracolsep{\fill}}r}
      \textbf{#1} & #2 \\
      \textit{\small#3} & \textit{\small #4} \\
    \end{tabular*}\vspace{-7pt}
}

\newcommand{\resumeSubSubheading}[2]{
    \item
    \begin{tabular*}{0.97\textwidth}{l@{\extracolsep{\fill}}r}
      \textit{\small#1} & \textit{\small #2} \\
    \end{tabular*}\vspace{-7pt}
}

\newcommand{\resumeProjectHeading}[2]{
    \item
    \begin{tabular*}{0.97\textwidth}{l@{\extracolsep{\fill}}r}
      \small#1 & #2 \\
    \end{tabular*}\vspace{-7pt}
}

\newcommand{\resumeSubItem}[1]{\resumeItem{#1}\vspace{-4pt}}

\renewcommand\labelitemii{$\vcenter{\hbox{\tiny$\bullet$}}$}

\newcommand{\resumeSubHeadingListStart}{\begin{itemize}[leftmargin=0.15in, label={}]}
\newcommand{\resumeSubHeadingListEnd}{\end{itemize}}
\newcommand{\resumeItemListStart}{\begin{itemize}}
\newcommand{\resumeItemListEnd}{\end{itemize}\vspace{-5pt}}

\begin{document}



\begin{center}
    {\LARGE Илья Балашов} \\ \vspace{2pt}
    \begin{multicols}{2}
    \begin{flushleft}
    \href{{https://github.com/ibalashov24/}}{GitHub: \ \  https://github.com/ibalashov24/} \\
    \href{{https://www.linkedin.com/in/ilya-balashov/}}{LinkedIn: https://www.linkedin.com/in/ilya-balashov/}
    \end{flushleft}
    
    \begin{flushright}
%    \href{{your personal websit link}}{my personal site}\\
    \href{mailto:{ilya.v.balashov@gmail.com}}{\textbf{ilya.v.balashov@gmail.com}}\\
    +7 (952) 398-32-88
    
    \end{flushright}
    \end{multicols}
\end{center}


%-----------EDUCATION-----------
\vspace{-2pt}
\section{Образование}
  \resumeSubHeadingListStart
      \resumeSubheading
      {Санкт-Петербургский Государственный Университет}{Сентябрь 2017 -- Июль 2021}
      {Математическое обеспечение и администрирование информационных систем}{Санкт-Петербург}
	\resumeItem{\textit{Математико-Механический факультет, Ср.балл 4.96/5.00\\Активность: актив Студсовета факультета, курирование первокурсников}}
  \resumeSubHeadingListEnd


%-----------EXPERIENCE-----------
\section{Опыт}
  \resumeSubHeadingListStart
  	\resumeSubheading
     {Разработчик, \href{https://itiviti.com}{Itiviti, a Broadridge business}}{Май 2021 -- настоящее время}{Разработка бэкенда Itiviti Tbricks, команда High Availability}{Санкт-Петербург}
     \resumeItemListStart
     	\resumeItem{Создал инструмент анализа логов продукта; с его помощью обнаружено 30+ багов}
     	\resumeItem{Разработал инструмент экспорта конфигурации и аудита продукта;\\ Консультировал 5+ клиентов по инструменту и его развёртыванию, получен положительный фидбек}
     	\resumeItem{Участвую во внедрении \href{https://www.consul.io/}{HashiCorp Consul} для обнаружения сервисов}
     	\resumeItem{Стек: C++17/STL, Google Test, Boost, многопоточность, Bash, Python, SQL}
     \resumeItemListEnd
  
    \resumeSubheading
      {Исследователь, JetBrains Research}{Октябрь 2019 -- Июнь 2021}
      {\href{https://research.jetbrains.org/groups/plt_lab/projects/pe-for-gpgpu/}{Специализация программ, Лаборатория Языковых Инструментов}}{Санкт-Петербург}
      \resumeItemListStart
        \resumeItem{Реализовал эксперименты по специализации программ с разными инструментами \\ (C++, CUDA/OpenCL, разные DSL и LLVM IR)}
        \resumeItem{Разработал бенчмарки различных алгоритмов на графах (пакет SuiteSparse GraphBLAS)}
        \resumeItem{Модифицировал инструмент для специализации на базе Clang}
        \resumeItem{Защитил диплом бакалавра по теме работы, участвовал в корпоративных конференциях}
    \resumeItemListEnd

  \resumeSubHeadingListEnd

%-----------PUBLICATIONS-----------


%-----------CERTIFICATIONS-----------
\section{Проекты}
 \resumeSubHeadingListStart
 	\resumeSubheading{Улучшение Android-приложения TRIK Gamepad }{Июнь 2019 -- Август 2019}
 	{\href{{https://play.google.com/store/apps/details?id=com.trikset.gamepad}}{Google Play}
 	}{}
 	\resumeItemListStart
 		\resumeItem{Разработал несколько новых экранов}
 		\resumeItem{Переделал сетевую часть приложения для поддержки UDP}
 		\resumeItem{Покрыл код различными типами тестов и настроил Circle CI}
 		\resumeItem{Мой код интегрирован в официальный релиз TRIK Gamepad}
 	\resumeItemListEnd
 	
 	\resumeSubheading{Тестирующая система для программ к роботам ТРИК}{Февраль 2019 -- Июнь 2019}
 	{\href{https://github.com/ibalashov24/semester\_4\_coursework}{Github}}
 	{}
 	\resumeItemListStart
 	\resumeItem{Интегрировал симулятор робота ТРИК с Travis CI}
 	\resumeItem{Реализовал возможность использовать симулятор как облачную тестирующую систему}
 	\resumeItem{Система была использована разработчиками ТРИК для тестирования различных программ проекта}
 	\resumeItemListEnd
 	
 	\resumeSubheading{Плагин Google Drive среды визуального программирования REAL.NET}{Сентябрь 2018 -- Декабрь 2018}
 	{\href{https://github.com/yurii-litvinov/REAL.NET/pull/61/}{Github}}{}
 	\resumeItemListStart
 	\resumeItem{Разработал плагин с функциональностью файлового менеджера на стеке .NET}
 	\resumeItem{Реализовал интеграцию плагина с Google Drive}
 	\resumeItem{Протестировал плагин на добровольцах, были получены положительные отзывы}
 	\resumeItemListEnd
 \resumeSubHeadingListEnd
 
 %-----------PROGRAMMING SKILLS-----------
 \section{Навыки}
 \begin{itemize}[leftmargin=0.15in, label={}]
 	\small{\item{
 			\textbf{Языки программирования и технологии}{:
 				\begin{itemize}
 					\item C++17, Google Test, SQL, Git, многопоточность, Linux (промышленный опыт)
 					\item Boost, C, Python, Bash, Docker, CUDA/OpenCL, стек .NET, Java/Kotlin (некоторый опыт)
 					\item Haskell, LLVM, x86 (знакомство)
 				\end{itemize} }

 			\textbf{Языки}{:\\ 
 				\begin{itemize}
 					\item Русский (родной)
 					\item Английский (B2, Upper Intermediate)
 				\end{itemize}}
 	}}
 \end{itemize}

\end{document}